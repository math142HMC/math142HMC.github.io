\documentclass[12pt]{article}
\usepackage[tmargin=1.25in,lmargin=1in,rmargin=1in,bmargin=1in,paper=letterpaper]{geometry}
\usepackage{amsmath,amssymb,amsthm,framed}
\usepackage{multirow,color,multicol}
\usepackage{fancyhdr,ifthen,lastpage}
\usepackage{verbatim,enumerate,cancel}
% --------------------------------------------------------------------
\newtheorem{lemma}{Lemma}
\newtheorem{thm}{Theorem}
\newtheorem{defn}{Definition}
\newtheorem{prop}{Proposition}

\theoremstyle{remark}
\newtheorem*{solution}{Solution}

\newcommand{\Z}{\mathbb{Z}}
\newcommand{\C}{\mathbb{C}}
\newcommand{\Q}{\mathbb{Q}}
\newcommand{\N}{\mathbb{N}}
\newcommand{\R}{\mathbb{R}}
\newcommand{\B}{\mathcal{B}}
\renewcommand{\P}{\mathcal{P}}
\renewcommand{\L}{\mathcal{L}}
\newcommand{\F}{\mathbf{F}}
\newcommand{\x}{\mathbf{x}}
\newcommand{\y}{\mathbf{y}}

\newcommand{\sm}{\setminus}
\newcommand{\es}{\emptyset}
\newcommand{\ol}{\overline}
\newcommand{\inv}{^{-1}}
\newcommand{\seq}[1]{\{ {#1}_n\}}
\newcommand{\ds}{\displaystyle}
\newcommand{\mbf}{\mathbf}
\renewcommand{\=}{&=&}
\newcommand{\<}{\langle}
\renewcommand{\>}{\rangle}

\newcommand{\bmat}{\begin{pmatrix}}
\newcommand{\emat}{\end{pmatrix}}
\newcommand{\beq}{\begin{eqnarray*}}
\newcommand{\eeq}{\end{eqnarray*}}

\renewcommand{\labelenumi}{(\alph{enumi})}

\DeclareMathOperator{\repart}{Re}
\DeclareMathOperator{\impart}{Im}
\DeclareMathOperator{\Arg}{Arg}
\DeclareMathOperator{\trace}{tr}
\DeclareMathOperator{\rk}{rank}
\DeclareMathOperator{\nullsp}{null}
\DeclareMathOperator{\range}{range}
\DeclareMathOperator{\vspan}{span}

% --------------------------------------------------------------------
\begin{document}
%
\newcommand{\deriv}[2]{\ensuremath{\frac{d{#1}}{d{#2}}}}
\newcommand{\dderiv}[2]{\ensuremath{\frac{d^2{#1}}{d{#2}^2}}}

%---------------------------------------------------------------------
\parindent=0in
%
\newcounter{probnum}
\newenvironment{problems}
   {\begin{list}{{\bf A\arabic{probnum}:}}%
       {\setlength\labelwidth{.25in}%
         \setlength\leftmargin{0in}%
         \setlength\itemsep{\parsep}%
         \usecounter{probnum}}}%
   {\end{list}}
%
% HEADER & FOOTER %
\pagestyle{fancy}
% HEADER %
\lhead{ Math 142}
\rhead{Prof. Gu}
\chead{\bf Linear Algebra Review Answer Key}
% FOOTER %
\lfoot{} 
\rfoot{}
\cfoot{\ifthenelse{\equal{\thepage}{\pageref{lastpage}}}{}{\footnotesize{{\color{red}(please turn over)}}}}
%%%%%


\begin{center}
\Large Linear Algebra Review
\end{center}
{\bf Due date:} \_\_\_\_\_\_\_\_\_\_\_\_\_\_

\begin{problems}


\item

For which of the following matrices are you \emph{guaranteed} a real diagonal form or no
real diagonal form at all without first determining the existence of an eigenbasis? Why?
%\begin{framed}
\begin{align*}
	A &= \bmat 5 & 0 & -1 \\ 0 & 3 & 3 \\ -1 & 3 & 0 \emat &
	B &= \bmat 2 & 5 & 3 \\ 0 & 3 & 0 \\ 0 & 0 & 2 \emat &
	C &= \bmat \frac{\sqrt{3}}{2} & -\frac{1}{2} \\ \frac{1}{2} & \frac{\sqrt{3}}{2} \emat \\
	D &= \bmat 0 & 1 \\ -1 & 0 \emat &
	E &= \bmat 1 & a \\ 0 & 1 \emat &
	F &= \bmat 1 & 3 \\ 2 & 2 \emat
\end{align*}
%\end{framed}

%\begin{solution}
	
%\end{solution}

\item 
Let $$A = \bmat 1 &  0 & -4 \\ 0 & 5 & 4 \\ -4 & 4 & 3 \emat.$$
\begin{enumerate}
	%\begin{framed}
	\item Find the eigenvalues and corresponding eigenvectors of $A$.
	%\end{framed}
	%\begin{solution}
	
	%\end{solution}
	
	%\begin{framed}
	\item Is $A$ similar to a diagonal matrix? If so, find a nonsingular matrix $P$ such that
	$P\inv A P$ is diagonal. Is $P$ unique? Explain.
	%\end{framed}
	%\begin{solution}
	
	%\end{solution}
	
	%\begin{framed}
	\item Find the eigenvalues of $A\inv$.
	%\end{framed}
	%\begin{solution}
	
	%\end{solution}
	
	%\begin{framed}
	\item Find the eigenvalues and corresponding eigenvectors of $A^2$.
	%\end{framed}
	%\begin{solution}
	
	%\end{solution}
\end{enumerate}



\item  
Let $L : \P_2 \to \P_2$ be defined by $$L(a+bt+ct^2) = (2a-c) + (a+b-c) t + c t^2.$$
\begin{enumerate}
	%\begin{framed}
	\item Find the matrix $A$ representing $L$ with respect to the standard basis of $\P_2$.
	%\end{framed}
	%\begin{solution}
	
	%\end{solution}
	
	%\begin{framed}
	\item Find all the eigenvalues of $A$. For each eigenvalue, find all eigenvectors 
	associated with that eigenvalue.
	%\end{framed}
	%\begin{solution}
	
	%\end{solution}
	
	%\begin{framed}
	\item Find a matrix $P$ such that $P\inv AP$ is diagonal.
	%\end{framed}
	%\begin{solution}
	
	%\end{solution}
	
	%\begin{framed}
	\item Find $A^n$ where $n$ is an integer. What is $L^{100}$?
	%\end{framed}
	%\begin{solution}
	
	%\end{solution}
\end{enumerate}




\item 
Let $A$ be an $n \times n$ real matrix.
\begin{enumerate}
	%\begin{framed}
	\item Prove that the coefficient of $\lambda^{n-1}$ in the characteristic polynomial
	of $A$ is given by $-\trace A$.
	%\end{framed}
	%\begin{solution}
	
	%\end{solution}
	
	%\begin{framed}
	\item Prove that $\trace A$ is the sum of the eigenvalues of $A$.
	%\end{framed}
	%\begin{solution}
	
	%\end{solution}
	
	%\begin{framed}
	\item Prove that the constant coefficient of the characteristic polynomial of $A$ is
	$\pm$ the product of the eigenvalues of $A$.
	%\end{framed}
	%\begin{solution}
	
	%\end{solution}
\end{enumerate}



\item 
Let $A$ be a $5 \times 5$ matrix. Suppose $A$ has distinct eigenvalues
$-1,1,-10,5,2$.
\begin{enumerate}
	%\begin{framed}
	\item What is $\det A$? What is $\trace A$?
	%\end{framed}
	%\begin{solution}
	
	%\end{solution}
	
	%\begin{framed}
	\item If $A$ and $B$ are similar, what is $\det B$? Why?
	%\end{framed}
	%\begin{solution}
	
	%\end{solution}
	
	%\begin{framed}
	\item Do you expect that all eigenvectors of $A$ are mutually orthogonal? Why?
	%\end{framed}
	%\begin{solution}
	
	%\end{solution}
\end{enumerate}



\item
\emph{This is an extra credit-type problem.}
Let $p_1(\lambda)$ be the characteristic polynomial of $A_{11}$ and $p_2(\lambda)$ the
characteristic polynomial of $A_{22}$. What is the characteristic polynomial of each of the
following partitioned matrices?
\beq
	A = \bmat A_{11} & 0 \\ 0 & A_{22} \emat \qquad
	B = \bmat A_{11} & A_{21} \\ 0 & A_{22} \emat
\eeq
%\begin{solution}

%\end{solution}




\item 
\begin{enumerate}
	%\begin{framed}
	\item Prove that similar matrices have the same eigenvalues.
	%\end{framed}
	%\begin{solution}
	
	%\end{solution}
	
	%\begin{framed}
	\item Let $\lambda_1, \lambda_2, \ldots, \lambda_k$ be distinct eigenvalues of a matrix
	$A$ with associated eigenvectors $x_1, x_2, \ldots, x_k$. Prove that $x_1, x_2, \ldots,
	x_k$ are linearly independent.
	%\end{framed}
	%\begin{solution}
	
	%\end{solution}
	
	%\begin{framed}
	\item Let $L:\R^n \to \R^n$ be a linear transformation defined by $L(X) = AX$. Let
	$V_\lambda = \{\xi \in \R^n \mid L(\xi) = \lambda \xi\}$. Prove $V_\lambda$ is a subspace
	of $\R^n$. (This subspace is called the eigenspace associated with $\lambda$.)
	%\end{framed}
	%\begin{solution}
	
	%\end{solution}
	
	%\begin{framed}
	\item Let $\lambda$ be an eigenvalue of $A$ with multiplicity $r$. Let $\dim V_\lambda
	= s$. Prove $s \le r$. (That is, the dimension of the eigenspace associated with 
	$\lambda$ is at most the multiplicity of $\lambda$.)
	%\end{framed}
	%\begin{solution}
	
	%\end{solution}
\end{enumerate}



\item
Let
\beq
	u = \bmat 1 \\ 0 \\ 1 \emat, \qquad 
	v = \bmat 1 \\ -1 \\ 0 \emat, \qquad 
	w = \bmat 2 \\ 2 \\ -\sqrt{6} \emat.
\eeq
\begin{enumerate}
	%\begin{framed}
	\item Find $\|u\|,\|v\|$. Find a unit vector in the direction of $u$.
	%\end{framed}
	%\begin{solution}
	
	%\end{solution}
	
	%\begin{framed}
	\item Find the distance between $v$ and $w$.
	%\end{framed}
	%\begin{solution}
	
	%\end{solution}
	
	%\begin{framed}
	\item Find angle between $u$ and $v$.
	%\end{framed}
	%\begin{solution}
	
	%\end{solution}
	
	%\begin{framed}
	\item Show that $v$ and $w$ are orthogonal.
	%\end{framed}
	%\begin{solution}
	
	%\end{solution}
\end{enumerate}



\item
\begin{enumerate}
	%\begin{framed}
	\item Prove the Cauchy-Schwarz Inequality: If $u$ and $v$ are any vectors in an inner 
	product space $V$, then $\langle u, v \rangle^2 \le \|u\|^2 \|v\|^2$.
	%\end{framed}
	%\begin{solution}
	
	%\end{solution}
	
	%\begin{framed}
	\item Consider $\R^n$ with the standard inner product. Let
	$u = (u_1, u_2, \ldots, u_n)$ and $v~=~(v_1, v_2, \ldots, v_n)$. Prove that 
	\beq
		\left(\sum_{i=1}^n u_i v_i \right)^2 \le \left(\sum_{i=1}^n u_i^2 \right)
		\left(\sum_{i=1}^n v_i^2 \right).
	\eeq
	%\end{framed}
	%\begin{solution}
	
	%\end{solution}
	
	%\begin{framed}
	\item Let $V$ be the vector space of all continuous real-valued functions on the
	unit interval $[0,1]$ with inner product $\<f,g\> = \int_0^1 f(t) g(t) \, dt$. Prove
	\beq
		\left(\int_0^1 f(t) g(t) \, dt \right)^2 \le \left(\int_0^1 f^2(t) \, dt \right)
		\left(\int_0^1 g^2(t) \, dt \right).
	\eeq
	%\end{framed}
	%\begin{solution}
	
	%\end{solution}
\end{enumerate}



\item
Let $C = [c_{ij}]$ be an $n \times n$ symmetric matrix and let $V$ be an $n$-dimensional
vector space with ordered basis $S = \{u_1, u_2, \ldots, u_n\}$. For $v = a_1 u_1 + a_2 u_2
+ \cdots + a_n u_n$ and $w = b_1 u_1 + b_2 u_2 + \cdots + b_n u_n$ in $V$, define
$$(v,w) = \sum_{i=1}^n \sum_{j=1}^n a_i c_{ij} b_j.$$ Prove that this defines an inner
product on $V$ if and only if $C$ is a positive-definite matrix.

%\begin{solution}
	
%\end{solution}



\item
Let $V$ be the vector space of all continuous functions on the interval $[-\pi, \pi]$. For $f$ and
$g$ in $V$, define $\<f,g\> = \int_{-\pi}^\pi f(t) g(t) \, dt$.
\begin{enumerate}
	%\begin{framed}
	\item Show that this defines an inner product on $V$.
	%\end{framed}
	%\begin{solution}
	
	%\end{solution}
	
	%\begin{framed}
	\item Show that the following set is an orthogonal set:
	$$\{1, \cos t, \sin t, \cos 2t, \sin 2t, \ldots, \cos nt, \sin nt, \ldots \}.$$
	%\end{framed}
	%\begin{solution}
	
	%\end{solution}
	
	%\begin{framed}
	\item Convert the above set into an orthonormal set.
	%\end{framed}
	%\begin{solution}
	
	%\end{solution}
\end{enumerate}



\item
A linear transformation $L:V \to V$, where $V$ is an $n$-dimensional Euclidean
space, is called \emph{orthogonal} if $\<Lv,Lw\> = \<v,w\>$.
\begin{enumerate}
	%\begin{framed}
	\item Let $A$ be an $n \times n$ matrix. Show that $A$ is orthogonal if and only if the
	columns (and rows) of $A$ form an orthonormal basis for $\R^n$.
	%\end{framed}
	%\begin{solution}
	
	%\end{solution}
	
	%\begin{framed}
	\item Let $S$ be an orthonormal basis for $V$ and let the matrix $A$ represent the 
	orthogonal linear transformation $L$ with respect to $S$. Prove that $A$ is an 
	orthogonal matrix.
	%\end{framed}
	%\begin{solution}
	
	%\end{solution}
	
	%\begin{framed}
	\item Prove that for any vectors $u, v \in \R^n$, $\<Lu,Lv\> = \<u,v\>$ if and only if for
	any $u \in \R^n$, $\|Lu\|=\|u\|$.
	%\end{framed}
	%\begin{solution}
	
	%\end{solution}
	
	%\begin{framed}
	\item Let $L:V \to V$ be an orthogonal linear transformation. Show that
	if $\lambda$ is an eigenvalue of $L$, then $|\lambda| = 1$.
	%\end{framed}
	%\begin{solution}
	
	%\end{solution}
\end{enumerate}



\item
Let $W$ be the subspace of the Euclidean space $\R^4$ with standard inner product with
basis $S = \{u_1, u_2, u_3\}$, where
\beq
	u_1 = \bmat 1 \\ 1 \\ 1 \\ 0 \emat, \qquad
	u_2 = \bmat -1 \\ 0 \\ -1 \\ 1 \emat, \qquad
	u_3 = \bmat -1 \\ 0 \\ 0 \\ -1 \emat.
\eeq
Transform $S$ to an orthonormal basis $T = \{w_1, w_2, w_3\}$ using the Gram-Schmidt
process.

%\begin{solution}
	
%\end{solution}



\item
\begin{enumerate}
	%\begin{framed}
	\item Let $$A = \bmat -1 & 3 & 3 \\ 3 & -1 & 3 \\ 3 & 3 & -1 \emat.$$ Find a $3 \times 3$ 
	matrix $P$ with $P \inv = P^T$ such that $P^T A P = D$, where $D$ is a $3 \times 3$ 
	diagonal matrix.
	%\end{framed}
	%\begin{solution}
	
	%\end{solution}
	
	%\begin{framed}
	\item (Extra credit) Show that all the eigenvalues of a real symmetric matrix are real 
	numbers.
	%\end{framed}
	%\begin{solution}
	
	%\end{solution}
	
	%\begin{framed}
	\item Show that if $A$ is a symmetric real matrix, then eigenvectors that belong to 
	distinct eigenvalues of $A$ are orthogonal.
	%\end{framed}
	%\begin{solution}
	
	%\end{solution}
	
	%\begin{framed}
	\item Prove that a symmetric matrix $A$ is positive-definite if and only if $A = P^T P$
	for a nonsingular matrix $P$.
	%\end{framed}
	%\begin{solution}
	
	%\end{solution}
	
	%\begin{framed}
	\item Prove that if the matrix $A$ is similar to a diagonal matrix, then $A$ is similar to
	$A^T$.
	%\end{framed}
	%\begin{solution}
	
	%\end{solution}
\end{enumerate}



\item
Consider two adjoining cells separated by a permeable membrane and suppose that a fluid
flows from the first cell to the second one at a rate (in milliliters per minute) that is numerically
equal to three times the volume (in milliliters) of the fluid in the first cell. It then flows out of
the second cell at a rate (in milliliters per minute) that is numerically equal to twice the volume
in the second cell. Let $x_1(t)$ and $x_2(t)$ denote the volumes of the fluid in the first and
second cells at time $t$, respectively. Assume that initially the first cell has 40 milliliters of
fluid, while the second one has 5 milliliters of fluid. Find the volume of fluid in each cell at
time $t$.

%\begin{solution}
	
%\end{solution}



\item
Consider a plant that can have red flowers (R), pink flowers (P), or white flowers (W),
depending upon the genotypes RR, RW, and WW. When we cross each of these genotypes
with a genotype RW, we obtain the transition matrix
$$M=\bmat 0.5 & 0.25 & 0.0 \\ 0.5 & 0.5 & 0.5 \\ 0.0 & 0.25 & 0.5 \emat.$$
Suppose that each successive generation is produced by crossing only with plants of $RW$
genotype. When the process reaches equilibrium, what percentage of the plants will have 
red, pink, or white flowers?

%\begin{solution}
	
%\end{solution}



\item
\begin{enumerate}
	%\begin{framed}
	\item Compute the eigenvalues and eigenvectors of $$A = \bmat 1 & 1 \\ 1 & 0 \emat.$$
	%\end{framed}
	%\begin{solution}
	
	%\end{solution}
	
	%\begin{framed}
	\item Verify that if $A = PBP\inv$ and $k$ is a positive integer, then $A^k = PB^kP\inv$.
	%\end{framed}
	%\begin{solution}
	
	%\end{solution}
	
	%\begin{framed}
	\item Using a hand calculator or MATLAB, compute $f_8$, $f_{12}$, and $f_{20}$, 
	where $f_n$ is the $n$th Fibonacci number, starting with $f_0 = f_1 = 1$.
	%\end{framed}
	%\begin{solution}
	
	%\end{solution}
\end{enumerate}



\item
Determine which of the given quadratic forms in three variables are equivalent:
\beq
	g_1(\mbf x) &=& x_1^2 + x_2^2+x_3^3 + 2 x_1 x_2 \\
	g_2(\mbf x) &=& 2x_2^2 + 2x_3^2 + 2 x_2 x_3 \\
	g_3(\mbf x) &=& 3x_2^2 - 3x_3^2 + 8 x_2 x_3 \\
	g_4(\mbf x) &=& 3x_2^2 + 3x_3^2 - 4 x_2 x_3.
\eeq

%\begin{solution}
	
%\end{solution}



\item
Which of the following matrices are positive-definite?
\beq
	A = \bmat 2 & 1 & 1 \\ 1 & 2 & 1 \\ 1 & 1 & 2 \emat, \qquad
	B = \bmat 3 & 2 \\ 2 & 5 \emat, \qquad
	C = \bmat 1 & 4 & 5 \\ 0 & 2 & 6 \\ 0 & 0 & 3 \emat, \qquad
	E = \bmat 1 & 3 \\ 3 & 5 \emat.
\eeq

%\begin{solution}
	
%\end{solution}



\item
Let $g(\mbf x) = 3x_1^2 - 3x_2^2 - 3x_3^2 + 4 x_2 x_3$ be a quadratic form in three
variables. 
\begin{enumerate}
	%\begin{framed}
	\item Find a quadratic form in the type given in the Principal Axis Theorem that is 
	equivalent to $g$. What is the rank of $g$? What is the signature of $g$?
	%\end{framed}
	%\begin{solution}
	
	%\end{solution}
	
	%\begin{framed}
	\item Identify the surface $g(\mbf x) = 9$.
	%\end{framed}
	%\begin{solution}
	
	%\end{solution}
\end{enumerate}





\end{problems}
\label{lastpage}
\end{document}






