\documentclass[12pt]{article}
\title{\rightline {\Huge {Due: Wednesday Sept 13}}}
\author{\LARGE {HMC\quad Math 142 \quad Fall 2017} 
\\ {Prof. Gu}  
\\ {\LARGE Problem Set 2}}
\date{Start this assignment before Sunday night!}
\usepackage{hyperref}

%height and width adjustments

\setlength{\topmargin}{-1cm}
\setlength{\textwidth}{18cm}
\setlength{\textheight}{23cm}
\setlength{\oddsidemargin}{-1cm}

\begin{document}



\maketitle

\section*{ Read: } 

\begin{itemize}
\item{Baby Do Carmo, Differential Geometry
    of Curves and Surfaces:  
Sections 1-3, 1-4, 1-5 and 1-6 of Chapter 1}
\item{Handout 2}
\item{Lecture Notes}
\end{itemize}

\section*{ Do: }
\paragraph{A: Problems on Reviewing Cross Products in $R^3$.}
\begin{itemize}
{\item a) Problem 2 on page 14, Section 1-4, Baby Do Carmo.}
{\item b) Problem 5 on page 14, Section 1-4, Baby Do Carmo.}
{\item c) Problem 11 on page 15, Section 1-4, Baby Do Carmo.}
{\item d) Problem 13 on page 16, Section 1-4, Baby Do Carmo.}

\end{itemize}

\medskip
\paragraph{B: Problems from Lectures}

\begin{itemize}
{\item a) Find the length of the curve obtained by 
intersecting the sphere $x^2 + y^2 + z^2 = 4 $ and the cylinder 
$(x-1)^2 + y^2 = 1$ in $R^3$. }
\end{itemize}

\paragraph{C: Other Problems}
\begin{itemize}
{\item a) Problem 1 on page 5, Section 1-2, Baby Do Carmo.}
{\item b) Problem 3 on page 5, Section 1-2, Baby Do Carmo.}
{\item c) Problem 4 on page 5, Section 1-2, Baby Do Carmo.}
{\item d) Problem 5 on page 5, Section 1-2, Baby Do Carmo.}
\end{itemize}

%\paragraph{D: Extra Credit Problems}
%\begin{itemize}
%\end{itemize}

\paragraph{D: Application}
\begin{itemize}
\item{Find your favorite big dataset (download from yahoo stock data). You may use some of the datasets which can be found under \url{https://www.aeaweb.org/articles?id=10.1257/aer.104.1.1.}}
\item{Use Polynomial Fitting or Gaussian Process or Spline Method to fit the data to get curves.}
\item {Use functional data analysis and techniques you learned about curves such as velocity, acceleration,length and curvature to find information from your big dataset.}

\end{itemize}

\end{document}







