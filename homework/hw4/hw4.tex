\documentclass[12pt]{article}
\title{\rightline {\Huge {Due: Wed. Sept. 27}}}
\author{\LARGE {HMC\quad Math 142 \quad Fall 2017} 
\\ {Prof. Gu}  
\\ {\LARGE Problem Set 4}}
\date{Start this assignment before Sunday night!}


%height and width adjustments

\setlength{\topmargin}{-1cm}
\setlength{\textwidth}{18cm}
\setlength{\textheight}{23cm}
\setlength{\oddsidemargin}{-1cm}
\usepackage{hyperref}
\usepackage{amssymb, amsmath}
\begin{document}



\maketitle

\section*{ Read: } 

\begin{itemize}
\item{Baby Do Carmo, Differential Geometry
    of Curves and Surfaces:  
Sections 2-1, 2-2, Chapter 2}
\item{Handout 5}
\item{Lecture Notes}
\end{itemize}

\section*{ Do: }
\paragraph{A: Problems on Reviewing of Rigid Motions in $R^3$.}
\begin{itemize}

{\item a) Show that the set of rigid motions $E(3)$ forms a group. 
(Later, we will see that  $E(3)$ is in fact a Lie group.)} 
 
\end{itemize}

\medskip
% \paragraph{B: Problems from Lectures}

% \begin{itemize}
% {\item a) Show that of all simple closed curves in the plane with 
% given length $l$, a circle bounds the largest area.}

% \end{itemize}

\paragraph{B: Other Problems}
\begin{itemize}
{\item a) Problem 2 on page 29, Section 1-6, Baby Do Carmo.}
%{\item b) Problem 1 on page 47, Section 1-7, Baby Do Carmo.}
%{\item c) Problem 2 on page 47, Section 1-7, Baby Do Carmo.}
{\item b) Problem 3 on page 65, Section 2-2, Baby Do Carmo.}
{\item c) Problem 5 on page 65, Section 2-2, Baby Do Carmo.}
{\item d) Problem 10 on page 66, Section 2-2, Baby Do Carmo.}
{\item e) Problem 16 on page 67, Section 2-2, Baby Do Carmo.}

\end{itemize}

\paragraph{C: Exterior Wedge Product from Class}
\begin{itemize}
{\item a) Look at the definition in the link \url{https://en.wikipedia.org/wiki/Exterior_algebra} and prove that
\begin{itemize}
    \item 
${\displaystyle \mathbf {u} \wedge \mathbf {v} \wedge \mathbf {w} =(u_{1}v_{2}w_{3}+u_{2}v_{3}w_{1}+u_{3}v_{1}w_{2}-u_{1}v_{3}w_{2}-u_{2}v_{1}w_{3}-u_{3}v_{2}w_{1})(\mathbf {e} _{1}\wedge \mathbf {e} _{2}\wedge \mathbf {e} _{3})} $

\item 
 ${\displaystyle \dim \Lambda ^{k}(V)={\binom {n}{k}}.}$
 
 \item In characteristic 0, the 2-vector $\alpha$ has rank p if and only if
${\displaystyle {\underset {p}{\underbrace {\alpha \wedge \cdots \wedge \alpha } }}\not =0}$ and $\underset{p+1}{\underbrace{\alpha\wedge\cdots\wedge\alpha}}= 0$


\end{itemize}

}




\end{itemize}

% \paragraph{D: Extra Credit Problems}
% \begin{itemize}
% {\item Give a different solution to B a).}
% \end{itemize}


\end{document}




