\documentclass[12pt]{article}
\title{\rightline {\Huge {Due: Wednesday, October 23}}}
\author{\LARGE {HMC\quad Math 142 \quad Midterm Exam Fall 2017} 
\\ Prof. Gu
}



%height and width adjustments

\setlength{\topmargin}{-1cm}
\setlength{\textwidth}{18cm}
\setlength{\textheight}{23cm}
\setlength{\oddsidemargin}{-1cm}

\begin{document}

\maketitle
\textbf{Note:} Official Exam Date is October 18. There is no class on that day and you are expected to work on this exam, but you are encouraged to start this project early.


You must write a paper consisting of both a theoretical part and an applied part.  The lengths of these two sections doesn't matter as long as they are both present, and that the total page length of the paper is at least 6-8 pages.  An acceptable length of the paper will be determined by your previous homework choice (please indicate below which one you had chosen). If you had chosen choice 1 you are required 8 pages, choice 2 required 6 pages, choice 3 required 7 pages. The paper needs to utilize concepts in differential geometry and manifold theory.  You are encouraged to look at published research articles, textbooks, or other online material/publications, as long as differential geometry remains the central focus.  You are more than welcome to present your own original research if you like as well.


% \section*{ Read: } 

% \begin{itemize}
% \item{Baby Do Carmo, Differential Geometry
%     of Curves and Surfaces:  
% Sections 2-4, 2-5, 2-6 and Section 5-10 on Abstract surfaces (starting 
% on page 425)} 
% \item{Handouts 8 and 9}
% \item{Lecture Notes}
% \end{itemize}

\section*{ Please circle one of the following for the homework 6 you had chosen}
\begin{enumerate}
    \item Choose a project of your preference. Write 2 pages of your work of the project that could be background sections, introduction, development or implementation of an algorithm or a model, some theorems, etc.  Identify at least one or a couple research articles. 
    
    \item Choose to work on homework problems. Work on part B, and choose 3 problems that interest you from part C to write up. Keys for part C are attached in resources.
    \item Choose half and half. Choose two problems from choice 2 (you can either choose 2 problems from part C or you can choose 1 problem on part C and works out the problem in part B ) and one page of work on choice 1
\end{enumerate}



% \medskip
% \paragraph{B: Problems from Lectures}

% \begin{itemize}
% % {\item a) Let $S$ be a subset of $R^3$.  Show that $S$ is 
% % regular surface if and only if S is locally diffeomorphic 
% % to $R^2$.}

% {\item a) Find five examples of regular surfaces such that each of 
% them can be represented as a surface of revolution.  Write down
% specifically for each example the generating curve, the rotation 
% axis, and the parameterization (as a map) for the surface (including
% the domain of the map).}

% \end{itemize}

% \paragraph{C: Problems to choose from}
% \begin{itemize}
% {\item a) Problem 10 on page 81, Section 2-3, Baby Do Carmo.}
% {\item b) Problem 9 on page 89, Section 2-4, Baby Do Carmo.}
% {\item c) Problem 15 on page 90, Section 2-4, Baby Do Carmo.}
% {\item d) Problem 18 on page 90, Section 2-4, Baby Do Carmo.}
% {\item e) Problem 1 on page 99, Section 2-5, Baby Do Carmo.}
% {\item f) Problem 3 on page 99, Section 2-5, Baby Do Carmo.}
% {\item g) Problem 9 on page 100, Section 2-5, Baby Do Carmo.}
% \end{itemize}


% \paragraph{D: Extra Credit Problems}
% \begin{itemize}
% {\item a) Let $T\subset R^3$ be a torus of revolution with center in 
% $(0,0,0)\in R^3$ and let $A(x,y,z)=(-x,-y,-z)$.
% Let $K$ be the quotient space of the torus $T$ by the equivalence 
% relation $p\sim A(p)$.  Can you tell what surface $K$ is? }
% {\item b) Show that $K$ is a differentiable 2-dimensional manifold.}
% {\item c) Show that $K$ is non orientable in two different ways.} 
% \end{itemize}

\end{document}













